\documentclass[a4paper,12pt]{article} % Можно заменить на report, если документ объемный

% Подключение пакетов
\usepackage[utf8]{inputenc}   % Поддержка UTF-8
\usepackage[russian]{babel}   % Русский язык
\usepackage{geometry}         % Настройки полей страницы
\geometry{a4paper, left=30mm, right=15mm, top=20mm, bottom=20mm}

\usepackage{setspace}         % Управление интервалами
\onehalfspacing               % Полуторный интервал

\usepackage{indentfirst}      % Красная строка в первом абзаце

\usepackage{graphicx}         % Работа с картинками
\usepackage{amsmath, amssymb} % Математические символы

% Заголовок документа
\title{Пояснительная записка к дипломному проекту "Разработка desktop-приложения для учёта месторасположения вещей в пространстве хранения"}
\author{Белградская Елена Валерьевна}
\date{\today}

\begin{document}

\maketitle

\section*{Введение}
Организация хранения вещей в каком-либо пространстве, будь то склад, гараж, квартира и т.д. не простая задача. 
Особенно сложно эффективно реализовывать такую задачу, если нет удобного, визуализированного 
представления месторасположения уже имеющихся вещей в пространстве хранения.  
Если рассмотреть организацию хранения вещей в квартире, то можно предположить не редской ситуацию, когда сложно вспомнить, выкинута или нет та или иная вещь. Человек забыл, где точно хранилась эта вещь, предполагаемые места 
хранения проверить на наличие или отсутствие вещи представляется проблематичным по тем или иным причинам. 
Если рассматривать учёт хранения товаров на складе, то важность строгого учета месторасположения, количества и т.д. товаров на складе очевидна.
В то время, как приложений для учета хранения товаров на складе достаточно много, то удобных приложений для учета хранения вещей в "небольших" пространствах таких как квартиры, дачного дома, гаража и т.д. не много.
В связи с этим разработка приложения для учёта месторасположения вещей в пространстве хранения представляется целесообразной задачей.
\par
Целью данной дипломной работы является "Разработка desktop-приложения для учёта месторасположения вещей в пространстве хранения".
\\
Для достижения поставленной цели необходимо выполнить следующие задачи:

\begin{itemize}
\item Исследовать предметную область. Формализовать предметную область.
\item Сформулировать функциональные требования.
\item Выбрать средства реализации.
\item Разработать и создать БД.
\item Разработать клиентскую часть.
\item Протестировать.
\end{itemize}

\section{Основная часть}
Основная часть может включать несколько подразделов:
\subsection{Описание проекта}
\subsection{Используемые технологии}
\subsection{Результаты и выводы}

\section*{Заключение}
Здесь подводятся итоги работы.

\begin{thebibliography}{9}
\bibitem{ref1} Автор. Название книги. Год.
\end{thebibliography}

\end{document}
